% Options for packages loaded elsewhere
\PassOptionsToPackage{unicode}{hyperref}
\PassOptionsToPackage{hyphens}{url}
%
\documentclass[
]{article}
\usepackage{amsmath,amssymb}
\usepackage{lmodern}
\usepackage{ifxetex,ifluatex}
\ifnum 0\ifxetex 1\fi\ifluatex 1\fi=0 % if pdftex
  \usepackage[T1]{fontenc}
  \usepackage[utf8]{inputenc}
  \usepackage{textcomp} % provide euro and other symbols
\else % if luatex or xetex
  \usepackage{unicode-math}
  \defaultfontfeatures{Scale=MatchLowercase}
  \defaultfontfeatures[\rmfamily]{Ligatures=TeX,Scale=1}
\fi
% Use upquote if available, for straight quotes in verbatim environments
\IfFileExists{upquote.sty}{\usepackage{upquote}}{}
\IfFileExists{microtype.sty}{% use microtype if available
  \usepackage[]{microtype}
  \UseMicrotypeSet[protrusion]{basicmath} % disable protrusion for tt fonts
}{}
\makeatletter
\@ifundefined{KOMAClassName}{% if non-KOMA class
  \IfFileExists{parskip.sty}{%
    \usepackage{parskip}
  }{% else
    \setlength{\parindent}{0pt}
    \setlength{\parskip}{6pt plus 2pt minus 1pt}}
}{% if KOMA class
  \KOMAoptions{parskip=half}}
\makeatother
\usepackage{xcolor}
\IfFileExists{xurl.sty}{\usepackage{xurl}}{} % add URL line breaks if available
\IfFileExists{bookmark.sty}{\usepackage{bookmark}}{\usepackage{hyperref}}
\hypersetup{
  pdftitle={Biol 432 Assignment 3},
  pdfauthor={Chenyang Wu},
  hidelinks,
  pdfcreator={LaTeX via pandoc}}
\urlstyle{same} % disable monospaced font for URLs
\usepackage[margin=1in]{geometry}
\usepackage{color}
\usepackage{fancyvrb}
\newcommand{\VerbBar}{|}
\newcommand{\VERB}{\Verb[commandchars=\\\{\}]}
\DefineVerbatimEnvironment{Highlighting}{Verbatim}{commandchars=\\\{\}}
% Add ',fontsize=\small' for more characters per line
\usepackage{framed}
\definecolor{shadecolor}{RGB}{248,248,248}
\newenvironment{Shaded}{\begin{snugshade}}{\end{snugshade}}
\newcommand{\AlertTok}[1]{\textcolor[rgb]{0.94,0.16,0.16}{#1}}
\newcommand{\AnnotationTok}[1]{\textcolor[rgb]{0.56,0.35,0.01}{\textbf{\textit{#1}}}}
\newcommand{\AttributeTok}[1]{\textcolor[rgb]{0.77,0.63,0.00}{#1}}
\newcommand{\BaseNTok}[1]{\textcolor[rgb]{0.00,0.00,0.81}{#1}}
\newcommand{\BuiltInTok}[1]{#1}
\newcommand{\CharTok}[1]{\textcolor[rgb]{0.31,0.60,0.02}{#1}}
\newcommand{\CommentTok}[1]{\textcolor[rgb]{0.56,0.35,0.01}{\textit{#1}}}
\newcommand{\CommentVarTok}[1]{\textcolor[rgb]{0.56,0.35,0.01}{\textbf{\textit{#1}}}}
\newcommand{\ConstantTok}[1]{\textcolor[rgb]{0.00,0.00,0.00}{#1}}
\newcommand{\ControlFlowTok}[1]{\textcolor[rgb]{0.13,0.29,0.53}{\textbf{#1}}}
\newcommand{\DataTypeTok}[1]{\textcolor[rgb]{0.13,0.29,0.53}{#1}}
\newcommand{\DecValTok}[1]{\textcolor[rgb]{0.00,0.00,0.81}{#1}}
\newcommand{\DocumentationTok}[1]{\textcolor[rgb]{0.56,0.35,0.01}{\textbf{\textit{#1}}}}
\newcommand{\ErrorTok}[1]{\textcolor[rgb]{0.64,0.00,0.00}{\textbf{#1}}}
\newcommand{\ExtensionTok}[1]{#1}
\newcommand{\FloatTok}[1]{\textcolor[rgb]{0.00,0.00,0.81}{#1}}
\newcommand{\FunctionTok}[1]{\textcolor[rgb]{0.00,0.00,0.00}{#1}}
\newcommand{\ImportTok}[1]{#1}
\newcommand{\InformationTok}[1]{\textcolor[rgb]{0.56,0.35,0.01}{\textbf{\textit{#1}}}}
\newcommand{\KeywordTok}[1]{\textcolor[rgb]{0.13,0.29,0.53}{\textbf{#1}}}
\newcommand{\NormalTok}[1]{#1}
\newcommand{\OperatorTok}[1]{\textcolor[rgb]{0.81,0.36,0.00}{\textbf{#1}}}
\newcommand{\OtherTok}[1]{\textcolor[rgb]{0.56,0.35,0.01}{#1}}
\newcommand{\PreprocessorTok}[1]{\textcolor[rgb]{0.56,0.35,0.01}{\textit{#1}}}
\newcommand{\RegionMarkerTok}[1]{#1}
\newcommand{\SpecialCharTok}[1]{\textcolor[rgb]{0.00,0.00,0.00}{#1}}
\newcommand{\SpecialStringTok}[1]{\textcolor[rgb]{0.31,0.60,0.02}{#1}}
\newcommand{\StringTok}[1]{\textcolor[rgb]{0.31,0.60,0.02}{#1}}
\newcommand{\VariableTok}[1]{\textcolor[rgb]{0.00,0.00,0.00}{#1}}
\newcommand{\VerbatimStringTok}[1]{\textcolor[rgb]{0.31,0.60,0.02}{#1}}
\newcommand{\WarningTok}[1]{\textcolor[rgb]{0.56,0.35,0.01}{\textbf{\textit{#1}}}}
\usepackage{graphicx}
\makeatletter
\def\maxwidth{\ifdim\Gin@nat@width>\linewidth\linewidth\else\Gin@nat@width\fi}
\def\maxheight{\ifdim\Gin@nat@height>\textheight\textheight\else\Gin@nat@height\fi}
\makeatother
% Scale images if necessary, so that they will not overflow the page
% margins by default, and it is still possible to overwrite the defaults
% using explicit options in \includegraphics[width, height, ...]{}
\setkeys{Gin}{width=\maxwidth,height=\maxheight,keepaspectratio}
% Set default figure placement to htbp
\makeatletter
\def\fps@figure{htbp}
\makeatother
\setlength{\emergencystretch}{3em} % prevent overfull lines
\providecommand{\tightlist}{%
  \setlength{\itemsep}{0pt}\setlength{\parskip}{0pt}}
\setcounter{secnumdepth}{-\maxdimen} % remove section numbering
\ifluatex
  \usepackage{selnolig}  % disable illegal ligatures
\fi

\title{Biol 432 Assignment 3}
\author{Chenyang Wu}
\date{2022/1/26}

\begin{document}
\maketitle

\hypertarget{project-info}{%
\subsection{Project Info}\label{project-info}}

\hypertarget{github-user-name-wuris}{%
\paragraph{\texorpdfstring{\textbf{GitHub user name}:
Wuris}{GitHub user name: Wuris}}\label{github-user-name-wuris}}

\hypertarget{date-2022126}{%
\paragraph{\texorpdfstring{\textbf{Date}:
2022/1/26}{Date: 2022/1/26}}\label{date-2022126}}

\hypertarget{github-link-httpsgithub.comwurisbiol432_assignment3.git}{%
\paragraph{\texorpdfstring{\textbf{GitHub Link}:
\url{https://github.com/Wuris/Biol432_Assignment3.git}}{GitHub Link: https://github.com/Wuris/Biol432\_Assignment3.git}}\label{github-link-httpsgithub.comwurisbiol432_assignment3.git}}

\hypertarget{load-the-fallopiadata.csv}{%
\subsubsection{Load the
FallopiaData.csv}\label{load-the-fallopiadata.csv}}

\begin{Shaded}
\begin{Highlighting}[]
\FunctionTok{setwd}\NormalTok{(}\StringTok{"E:/Underguaduate/4th Fourth year/BIOL 432/Week 3/Biol432\_Assignment3"}\NormalTok{)}
\NormalTok{A3Data }\OtherTok{\textless{}{-}} \FunctionTok{read.csv}\NormalTok{(}\StringTok{"InputData/FallopiaData.csv"}\NormalTok{)}
\end{Highlighting}
\end{Shaded}

\hypertarget{load-some-packages}{%
\subsubsection{Load some packages}\label{load-some-packages}}

\begin{Shaded}
\begin{Highlighting}[]
\FunctionTok{library}\NormalTok{(dplyr)}
\end{Highlighting}
\end{Shaded}

\begin{verbatim}
## 
## 载入程辑包:'dplyr'
\end{verbatim}

\begin{verbatim}
## The following objects are masked from 'package:stats':
## 
##     filter, lag
\end{verbatim}

\begin{verbatim}
## The following objects are masked from 'package:base':
## 
##     intersect, setdiff, setequal, union
\end{verbatim}

\hypertarget{remove-rows-with-total-biomass-60}{%
\subsubsection{Remove rows with `Total' biomass \textless{}
60}\label{remove-rows-with-total-biomass-60}}

\begin{Shaded}
\begin{Highlighting}[]
\NormalTok{A3Data\_New }\OtherTok{\textless{}{-}}\NormalTok{ A3Data }\SpecialCharTok{\%\textgreater{}\%}
  \FunctionTok{filter}\NormalTok{(Total }\SpecialCharTok{\textgreater{}=} \DecValTok{60}\NormalTok{)}
\end{Highlighting}
\end{Shaded}

\hypertarget{reorder-the-columns-so-that-they-are-in-the-order-total-taxon-scenario-nutrients-and-remove-the-other-columns}{%
\subsubsection{Reorder the columns so that they are in the order:
`Total', `Taxon', `Scenario', `Nutrients', and remove the other
columns}\label{reorder-the-columns-so-that-they-are-in-the-order-total-taxon-scenario-nutrients-and-remove-the-other-columns}}

\begin{Shaded}
\begin{Highlighting}[]
\NormalTok{A3Data\_New }\OtherTok{\textless{}{-}}\NormalTok{ A3Data\_New }\SpecialCharTok{\%\textgreater{}\%} 
  \FunctionTok{select}\NormalTok{(Total, Taxon, Scenario, Nutrients)}
\end{Highlighting}
\end{Shaded}

\hypertarget{make-a-new-column-totalg-which-converts-the-total-column-from-mg-to-grams-and-replace-total-with-totalg-and-add-it-to-the-dataset.}{%
\subsubsection{Make a new column TotalG, which converts the `Total'
column from mg to grams AND replace Total with TotalG, and add it to the
dataset.}\label{make-a-new-column-totalg-which-converts-the-total-column-from-mg-to-grams-and-replace-total-with-totalg-and-add-it-to-the-dataset.}}

\begin{Shaded}
\begin{Highlighting}[]
\NormalTok{A3Data\_New}\SpecialCharTok{$}\NormalTok{TotalG }\OtherTok{=}\NormalTok{ A3Data\_New}\SpecialCharTok{$}\NormalTok{Total}\SpecialCharTok{*}\FloatTok{0.001}

\NormalTok{A3Data\_New }\OtherTok{\textless{}{-}}\NormalTok{ A3Data\_New }\SpecialCharTok{\%\textgreater{}\%} 
  \FunctionTok{select}\NormalTok{(TotalG, Taxon, Scenario, Nutrients)}
\end{Highlighting}
\end{Shaded}

\hypertarget{write-a-custom-function-that-will-take-two-inputs-from-the-user-1.-a-vector-of-data-to-process-e.g.-column-from-a-data.frame-object-and-2.-a-string-that-defines-what-calculation-to-perform.}{%
\subsubsection{Write a custom function that will take two inputs from
the user: 1. a vector of data to process (e.g.~column from a data.frame
object) and 2. a string that defines what calculation to
perform.}\label{write-a-custom-function-that-will-take-two-inputs-from-the-user-1.-a-vector-of-data-to-process-e.g.-column-from-a-data.frame-object-and-2.-a-string-that-defines-what-calculation-to-perform.}}

\hypertarget{if-string-2-is-average-then-calculate-the-average-value-for-the-column-named-in-vector-1}{%
\paragraph{if string \#2 is ``Average'' then calculate the average value
for the column named in vector
\#1}\label{if-string-2-is-average-then-calculate-the-average-value-for-the-column-named-in-vector-1}}

\hypertarget{if-string-2-is-sum-then-calculate-the-sum-of-values-for-the-column-named-in-vector-1}{%
\paragraph{if string \#2 is ``Sum'' then calculate the sum of values for
the column named in vector
\#1}\label{if-string-2-is-sum-then-calculate-the-sum-of-values-for-the-column-named-in-vector-1}}

\hypertarget{if-string-2-is-observations-then-count-the-number-of-observed-values-for-the-column-named-in-vector-1}{%
\paragraph{if string \#2 is ``Observations'' then count the number of
observed values for the column named in vector
\#1}\label{if-string-2-is-observations-then-count-the-number-of-observed-values-for-the-column-named-in-vector-1}}

\hypertarget{if-string-2-is-anything-else-then-output-an-error-to-the-user}{%
\paragraph{if string \#2 is anything else, then output an error to the
user}\label{if-string-2-is-anything-else-then-output-an-error-to-the-user}}

\begin{Shaded}
\begin{Highlighting}[]
\NormalTok{my\_custom\_function }\OtherTok{\textless{}{-}} \ControlFlowTok{function}\NormalTok{(x, y)\{}
  \ControlFlowTok{if}\NormalTok{(y }\SpecialCharTok{==} \StringTok{"Average"}\NormalTok{)\{}
    \FunctionTok{return}\NormalTok{(}\FunctionTok{mean}\NormalTok{(x))}
\NormalTok{  \} }\ControlFlowTok{else} \ControlFlowTok{if}\NormalTok{(y }\SpecialCharTok{==} \StringTok{"Sum"}\NormalTok{)\{}
    \FunctionTok{return}\NormalTok{(}\FunctionTok{sum}\NormalTok{(x))}
\NormalTok{  \} }\ControlFlowTok{else} \ControlFlowTok{if}\NormalTok{(y }\SpecialCharTok{==} \StringTok{"Observations"}\NormalTok{)\{}
    \FunctionTok{return}\NormalTok{(}\FunctionTok{length}\NormalTok{(x))}
\NormalTok{  \} }\ControlFlowTok{else}\NormalTok{ \{}
    \FunctionTok{return}\NormalTok{(}\FunctionTok{print}\NormalTok{(}\StringTok{"Error!"}\NormalTok{))}
\NormalTok{  \}}
\NormalTok{\}}
\end{Highlighting}
\end{Shaded}

\hypertarget{write-some-r-code-that-uses-your-function-to-count-the-total-number-of-observations-in-the-taxon-column.}{%
\subsubsection{Write some R code that uses your function to count the
total number of observations in the `Taxon'
column.}\label{write-some-r-code-that-uses-your-function-to-count-the-total-number-of-observations-in-the-taxon-column.}}

\begin{Shaded}
\begin{Highlighting}[]
\FunctionTok{my\_custom\_function}\NormalTok{(A3Data\_New}\SpecialCharTok{$}\NormalTok{Taxon, }\StringTok{"Observations"}\NormalTok{)}
\end{Highlighting}
\end{Shaded}

\begin{verbatim}
## [1] 45
\end{verbatim}

\hypertarget{write-some-r-code-that-uses-your-function-to-calculate-the-average-totalg-for-each-of-the-two-nutrient-concentrations.}{%
\subsubsection{Write some R code that uses your function to calculate
the average TotalG for each of the two Nutrient
concentrations.}\label{write-some-r-code-that-uses-your-function-to-calculate-the-average-totalg-for-each-of-the-two-nutrient-concentrations.}}

\begin{Shaded}
\begin{Highlighting}[]
\NormalTok{A3Data\_New }\SpecialCharTok{\%\textgreater{}\%} 
  \FunctionTok{group\_by}\NormalTok{(Nutrients) }\SpecialCharTok{\%\textgreater{}\%} 
  \FunctionTok{summarise}\NormalTok{(}\AttributeTok{Average =} \FunctionTok{my\_custom\_function}\NormalTok{(TotalG, }\StringTok{"Average"}\NormalTok{))}
\end{Highlighting}
\end{Shaded}

\begin{verbatim}
## # A tibble: 2 x 2
##   Nutrients Average
##   <chr>       <dbl>
## 1 high       0.0665
## 2 low        0.0641
\end{verbatim}

\hypertarget{write-i.e.-save-the-new-data-to-a-file-called-wrangleddata.csv-in-the-output-folder.}{%
\subsubsection{Write (i.e.~save) the new data to a file called
``WrangledData.csv'' in the Output
folder.}\label{write-i.e.-save-the-new-data-to-a-file-called-wrangleddata.csv-in-the-output-folder.}}

\begin{Shaded}
\begin{Highlighting}[]
\FunctionTok{write.csv}\NormalTok{(A3Data\_New, }\StringTok{"./Output/WrangledData.csv"}\NormalTok{) }
\end{Highlighting}
\end{Shaded}

\hypertarget{test-your-script-for-errors}{%
\subsubsection{Test your script for
errors}\label{test-your-script-for-errors}}

\begin{Shaded}
\begin{Highlighting}[]
\NormalTok{Test }\OtherTok{\textless{}{-}} \FunctionTok{c}\NormalTok{(}\DecValTok{1}\NormalTok{, }\DecValTok{2}\NormalTok{, }\DecValTok{3}\NormalTok{, }\DecValTok{4}\NormalTok{, }\DecValTok{5}\NormalTok{, }\DecValTok{6}\NormalTok{, }\DecValTok{7}\NormalTok{, }\DecValTok{8}\NormalTok{, }\DecValTok{9}\NormalTok{, }\DecValTok{10}\NormalTok{)}

\FunctionTok{my\_custom\_function}\NormalTok{(Test, }\StringTok{"Average"}\NormalTok{)}
\end{Highlighting}
\end{Shaded}

\begin{verbatim}
## [1] 5.5
\end{verbatim}

\begin{Shaded}
\begin{Highlighting}[]
\FunctionTok{my\_custom\_function}\NormalTok{(Test, }\StringTok{"Sum"}\NormalTok{)}
\end{Highlighting}
\end{Shaded}

\begin{verbatim}
## [1] 55
\end{verbatim}

\begin{Shaded}
\begin{Highlighting}[]
\FunctionTok{my\_custom\_function}\NormalTok{(Test, }\StringTok{"Observations"}\NormalTok{)}
\end{Highlighting}
\end{Shaded}

\begin{verbatim}
## [1] 10
\end{verbatim}

\begin{Shaded}
\begin{Highlighting}[]
\FunctionTok{my\_custom\_function}\NormalTok{(Test, }\StringTok{"Anything else, for error"}\NormalTok{)}
\end{Highlighting}
\end{Shaded}

\begin{verbatim}
## [1] "Error!"
\end{verbatim}

\end{document}
